\documentclass[12pt]{article}
\usepackage[utf8]{inputenc}
\usepackage{geometry}
\geometry{a4paper, margin=2cm}
\usepackage{hyperref}
\usepackage{longtable}
\usepackage{graphicx}

\title{Auditoria Técnica do Sistema \textit{tarefas2}}
\author{Relatório Gerado automaticamente}
\date{\today}

\begin{document}

\maketitle

\section{Visão Geral}
O sistema \textit{tarefas2} é um gerenciador de tarefas acadêmicas desenvolvido em C utilizando CGI. A arquitetura é modular, focada em usuários com diferentes papéis (aluno, professor e administrador). O sistema também utiliza bibliotecas próprias para banco de dados, manipulação de arquivos e sessões.

\section{Arquitetura do Sistema}
A estrutura possui módulos separados por função:
\begin{itemize}
    \item Autenticação e sessão
    \item Papéis de usuário (aluno, professor, administrador)
    \item Configuração e correção de tarefas
    \item Banco interno de dados
\end{itemize}

\subsection{Fluxo de Acesso}
\begin{verbatim}
Usuário acessa /tarefas.cgi
   ├─ Verifica sessão
   ├─ Se não logado: redireciona para login
   └─ Se logado: identifica papel do usuário
        ├─ aluno → /html/papeis/aluno/
        ├─ professor → /html/papeis/professor/
        └─ admin → /html/papeis/adm/
\end{verbatim}

\section{Bibliotecas Detectadas}
\begin{longtable}{|l|p{9cm}|}
\hline
\textbf{Biblioteca} & \textbf{Função} \\
\hline
bib/cgi/login & Sistema de login e autenticação \\ \hline
bib/cgi/papeis & Define o papel do usuário \\ \hline
bib/bd/tarefasbd.h & Banco de dados de tarefas \\ \hline
bib/imagem.h & Upload/exibição de imagens \\ \hline
bib/diretorio.h & Manipulação de arquivos no servidor \\ \hline
bib/str/* & Tratamento de strings e HTML \\ \hline
\end{longtable}

\section{Auditoria Técnica}
Foram analisados pontos críticos:

\subsection{Pontos Fortes}
\begin{itemize}
    \item Estrutura modular
    \item Independência de banco externo
    \item Separação clara de papéis
    \item Compatível com ambientes de ensino
\end{itemize}

\subsection{Riscos e Fragilidades}
\begin{itemize}
    \item Segurança de login e sessões deve ser revisada
    \item Possível vulnerabilidade em \textit{upload} de arquivos
    \item Ausência de logs e tratamento de erro padronizado
    \item Falta de documentação por módulo
    \item Dependência de CGI pode ser um gargalo de performance
\end{itemize}

\section{Recomendações Técnicas}
\begin{longtable}{|p{7cm}|p{7cm}|}
\hline
\textbf{Recomendação} & \textbf{Benefício} \\ \hline
Migrar para FastCGI ou PHP & Aumento de performance \\ \hline
Utilizar SQLite ou MySQL & Escalabilidade \\ \hline
Criar API REST em C ou Python & Integração com frontend moderno \\ \hline
Desenvolver UI em Vue ou React & Melhor experiência de usuário \\ \hline
Criar sistema de logs e erros & Melhor depuração \\ \hline
\end{longtable}

\section{Conclusão}
O sistema tem potencial acadêmico e pode ser transformado em uma plataforma moderna com poucas adaptações estruturais. Há pontos fortes na modularidade, mas melhorias são necessárias para produção e segurança.

\end{document}
